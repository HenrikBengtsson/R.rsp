%\VignetteIndexEntry{R packages: LaTeX vignettes}
%\VignetteKeyword{R, package, vignette, LaTeX}
%\VignetteEngine{R.rsp::tex}

\documentclass[12pt]{article}
\usepackage{xspace}
\usepackage{alltt}

% Adjust margins
\addtolength{\oddsidemargin}{-0.5in}	
\addtolength{\evensidemargin}{-0.5in}	
\addtolength{\textwidth}{1in}
\addtolength{\topmargin}{-0.5in}	
\addtolength{\textheight}{1in}

\newcommand{\keywords}[1]{\footnotesize{\textbf{Keywords: }#1}\xspace}
\newcommand{\pkg}[1]{\textsl{#1}\xspace}
\newcommand{\file}[1]{\textsl{#1}\xspace}
\newcommand{\code}[1]{\texttt{#1}\xspace}
\newcommand{\bs}{$\backslash$}

\newenvironment{rspVerbatim}{\vspace{-\parskip}\begin{alltt}\color{blue}}{\end{alltt}}
\newenvironment{escapeRspVerbatim}{\vspace{-\parskip}\begin{alltt}}{\end{alltt}}


\title{R packages: LaTeX vignettes}
\author{Henrik Bengtsson}
\date{2014-10-03}

\begin{document}

\maketitle

To include a PDF vignette that is compiled from a plain LaTeX file,
all you need is the LaTeX file with LaTeX comments containing
meta directives to R such that it will be listed as a vignette in the
package.  For instance, this PDF document was compiled from LaTeX file:
\begin{enumerate}
 \item \code{vignettes/R\_packages-LaTeX\_vignettes.tex}
\end{enumerate}
which contains the following meta directives at the top of the file:
\begin{verbatim} 
  %\VignetteIndexEntry{R packages: LaTeX vignettes}
  %\VignetteKeyword{R package, vignette, LaTeX}
  %\VignetteEngine{R.rsp::tex}
\end{verbatim}
As for any type of (non-Sweave) package vignette, don't forget to specify:
\begin{verbatim}
Suggests: R.rsp
VignetteBuilder: R.rsp
\end{verbatim}
in your package's DESCRIPTION file.  That's all it takes to include a
LaTeX file that will be compiled into a PDF vignette as part of the
package installation.
\end{document}
